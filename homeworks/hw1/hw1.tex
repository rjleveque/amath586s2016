\documentclass[10pt]{article}

\usepackage{graphicx}
\usepackage{amsmath,amsfonts,amssymb}

\usepackage{hyperref}  % for urls and hyperlinks


\setlength{\textwidth}{6.2in}
\setlength{\oddsidemargin}{0.3in}
\setlength{\evensidemargin}{0in}
\setlength{\textheight}{8.9in}
\setlength{\voffset}{-1in}
\setlength{\headsep}{26pt}
\setlength{\parindent}{0pt}
\setlength{\parskip}{5pt}



\input{../latex/macros.tex}  % input some useful macros

\begin{document}

% header:
\hfill\vbox{\hbox{AMath 586 / ATM 581}
\hbox{Homework \#1}\hbox{Due Thursday, April 7, 2016}}

\vskip 5pt

Homework is due to Canvas by 11:00pm PDT on the due date.

To submit, see \url{https://canvas.uw.edu/courses/1038268/assignments/3220198}


%--------------------------------------------------------------------------
\vskip 1cm
\hrule
{\bf Problem 1}


Prove that the ODE 
\[
u'(t) = \cos(t^2 + u(t)^2), \quad \mbox{for}~t \geq 0
\]
has a unique solution for all time from any initial value $u(0)=\eta$.


% uncomment the next two lines if you want to insert solution...
%\vskip 1cm
%{\bf Solution:}

% insert your solution here!

%--------------------------------------------------------------------------
\vskip 1cm
\hrule
{\bf Problem 2}

\begin{enumerate} 
\item
Use Duhamel's Principle to solve the ODE
\begin{equation}\label{ode1}
u'(t) = \lambda (u(t)-\cos(t)) - \sin(t)
\end{equation} 
with initial condition $u(t_0) = \eta$, for general real numbers $\lambda$ and
$\eta$.  

\item
Let $\lambda = -0.5$.  On a single graph, use Python to plot $u(t)$ 
(on the time interval $0 \leq t \leq 4\pi$)
for different choices of
initial condition $u(0) = \eta = -1,~-0.8,~-0.6,~\ldots,1.6,~1.8,~2.$ 

\item Do the same for $\lambda = -5$.
\end{enumerate} 

% uncomment the next two lines if you want to insert solution...
%\vskip 1cm
%{\bf Solution:}

% insert your solution here!

%--------------------------------------------------------------------------
\vskip 1cm
\hrule
{\bf Problem 3}

Consider the third-order ODE
\begin{equation*}
\begin{split}
&v'''(t) - v''(t) - 2v'(t) = 0\\
&v(0) = 12, \quad v'(0) = -8, \quad v''(0) = 14.
\end{split}
\end{equation*}

\begin{enumerate} 
\item Solve this equation by using the fact that a linear ODE of this form has
solutions of the form $v(t) = e^{rt}$ for certain values of $r$.  Plugging this
Ansatz into the equation shows that $r$ must be a root of a cubic equation.  There
are three distinct roots and hence three linearly independent solutions of this
form.  Find the proper linear combination of these to find the solution that also
satisfies the three initial conditions.

\item Solve this equation in a different way: rewrite it as a first-order system of
three equations of the form $u'(t) = Au(t)$ where $u(t)\in\reals^3$ and $A$ is a
$3\times 3$ matrix, with suitable initial conditions $u(0) = \eta \in \reals^3$.
Then compute the matrix exponential based on the eigenvalues and eigenvectors of
$A$ in order to find $u(t) = \exp(At)\eta$.  (See Section D.3 in the text.)
Confirm that the solution agrees with what you got before.  

\item Remind yourself what
the ``companion matrix'' for a polynomial is, and say why this is relevant to
relating the two solution techniques above.  See Section D.2.1 in the text for a
discussion of a similar technique for solving linear difference equations
that we will use to analyze certain numerical methods.

\item
Determine the best possible Lipschitz constant for this system in the max-norm
$\|\cdot\|_\infty$ and the 1-norm $\|\cdot\|_1$. (See Appendix A.3.)

\item Solve the first-order system you derived above
using the Python function {\tt odeint} from the
{scipy.integrate} module, with output times {\tt t = linspace(0,2,51)}.  Plot
this solution as points on top of the true solution you computed above to show that
you have done this correctly.
\end{enumerate} 


% uncomment the next two lines if you want to insert solution...
%\vskip 1cm
%{\bf Solution:}

% insert your solution here!



%--------------------------------------------------------------------------
\vskip 1cm
\hrule
{\bf Problem 4}

Compute the leading term in the local truncation error of the following
methods:
\begin{enumerate}
\item the trapezoidal method (5.22),
\item the 2-step Adams-Bashforth method,
\item the Runge-Kutta method (5.32).
\end{enumerate} 


% uncomment the next two lines if you want to insert solution...
%\vskip 1cm
%{\bf Solution:}

% insert your solution here!


%--------------------------------------------------------------------------
\vskip 1cm
\hrule
{\bf Problem 5}

Determine the coefficients $\beta_0,~\beta_1,~\beta_2$ for the third
order, 2-step Adams-Moulton method.  Do this in two different ways:
\begin{enumerate} 
 \item Using the expression for the local truncation error in Section 5.9.1,
 \item Using the relation
 \[
 u(t_{n+2}) = u(t_{n+1}) + \int_{t_{n+1}}^{t_{n+2}}\,f(u(s))\,ds.
 \]
 Interpolate  a quadratic polynomial $p(t)$ through the three values
 $f(U^n),~f(U^{n+1})$ and $f(U^{n+2})$ and then integrate this polynomial
 exactly to obtain the formula.  The coefficients of the polynomial will
 depend on the three values $f(U^{n+j})$.   It's easiest to use the
 ``Newton form'' of the interpolating polynomial and consider the three
times $t_n=-k$, $t_{n+1}=0$, and $t_{n+2}=k$ so that $p(t)$ has the form
\[
p(t) = A + B(t+k) + C(t+k)t
\]
where $A,~B$, and $C$ are the appropriate divided differences based on the
data.  Then integrate from $0$ to $k$.   (The method has the same
coefficients at any time, so this is valid.)
\end{enumerate}


% uncomment the next two lines if you want to insert solution...
%\vskip 1cm
%{\bf Solution:}

% insert your solution here!

%--------------------------------------------------------------------------



\end{document}

