\documentclass[10pt]{article}

\usepackage{graphicx}
\usepackage{amsmath,amsfonts,amssymb}

\usepackage{hyperref}  % for urls and hyperlinks


\setlength{\textwidth}{6.2in}
\setlength{\oddsidemargin}{0.3in}
\setlength{\evensidemargin}{0in}
\setlength{\textheight}{8.9in}
\setlength{\voffset}{-1in}
\setlength{\headsep}{26pt}
\setlength{\parindent}{0pt}
\setlength{\parskip}{5pt}



\input{../latex/macros.tex}  % input some useful macros

\begin{document}

% header:
\hfill\vbox{\hbox{AMath 586 / ATM 581}
\hbox{Homework \#3}\hbox{Due Thursday, April 28, 2016}}

\vskip 5pt

Homework is due to Canvas by 11:00pm PDT on the due date.

To submit, see
\url{https://canvas.uw.edu/courses/1038268/assignments/3262739}



%--------------------------------------------------------------------------
\vskip 1cm
\hrule
{\bf Problem 1}

Consider the implicit Runge-Kutta method
\begin{equation}\label{ex1}
\begin{split}
U^* &= U^n + \frac{k}{2} f\large(U^*, t_n + k/2\large),\\
U^{n+1} &= U^n + k f\large(U^*, t_n + k/2\large).
\end{split}
\end{equation} 
The first step is Backward Euler to determine an approximation to the value
at the midpoint in time and the second step is the midpoint method using
this value.

\begin{enumerate}
\item Determine the order of accuracy of this method.
\item Plot the region of absolute stability.
\item Is this method A-stable?  Is it L-stable?
\end{enumerate}

% uncomment the next two lines if you want to insert solution...
%\vskip 1cm
%{\bf Solution:}

% insert your solution here!


%--------------------------------------------------------------------------
\vskip 1cm
\hrule
{\bf Problem 2}


Plot the stability region for the TR-BDF2 method (8.6).   
You can start with the code in the
notebook {\tt Stability\_Regions\_onestep.ipynb}.

Observe that the method is A-stable and show by analyzing $R(z)$  that it is
also L-stable.

% uncomment the next two lines if you want to insert solution...
%\vskip 1cm
%{\bf Solution:}

% insert your solution here!


%--------------------------------------------------------------------------
\vskip 1cm
\hrule
{\bf Problem 3}


Let $g(x)=0$ represent a system of $s$ nonlinear equations in $s$ unknowns,
so $x\in\reals^s$ and $g: \reals^s \goto \reals^s$.  A vector $\bar
x\in\reals^s$ is a {\em fixed point} of $g(x)$ if 
\begin{equation}\label{a}
\bar x = g(\bar x).
\end{equation}
One way to attempt to compute $\bar x$ is with {\em fixed point iteration}:
from some starting guess $x^0$, compute
\begin{equation}\label{b}
x^{j+1} = g(x^j)
\end{equation}
for $j=0,~1,~\ldots$.

\begin{enumerate}
\item Show that if there exists a norm $\|\cdot\|$ such that $g(x)$ is
Lipschitz continuous with constant $L<1$ in a neighborhood of $\bar x$, then
fixed point iteration converges from any starting value in this
neighborhood.
{\bf Hint:} Subtract equation \eqn{a} from \eqn{b}.

\item Suppose $g(x)$ is differentiable and let $g'(x)$ be the $s\times s$
Jacobian matrix.  Show that if the condition of part (a) holds then
$\rho(g'(\bar x)) < 1$, where $\rho(A)$ denotes the spectral radius of a
matrix.

\newpage
\item Consider a predictor-corrector method (see Section 5.9.4) consisting
of forward Euler as the predictor and backward Euler as the corrector, and
suppose we make $N$ correction iterations, i.e., we set
\begin{tabbing}
xxxxxxxxx\=xxxx\=\kill\\
\>$\hat U^0 = U^n + kf(U^n)$\\
\>for $j = 0,~1,~\ldots,~N-1$\\
\>\>$\hat U^{j+1} = U^n + kf(\hat U^j)$\\
\>\>end\\
\>$U^{n+1} = \hat U^N$.
\end{tabbing}
Note that this can be interpreted as a fixed point iteration for solving the
nonlinear equation
\[
\unp = U^n + kf(\unp)
\]
of the backward Euler method.  Since the backward Euler method is implicit
and has a stability region that includes the entire left half plane, as
shown in Figure 7.1(b), one might hope that this predictor-corrector method
also has a large stability region.

Plot the stability region $S_N$ of this method for $N=2,~5,~10,~20,~50$ 
and observe that in fact the stability region does not grow much in size.

\item Using the result of part (b), show that the fixed point iteration
being used in the predictor-corrector method of part (c) can only be
expected to converge if $|k\lambda| < 1$ for all eigenvalues $\lambda$ of
the
Jacobian matrix $f'(u)$.  

\item Based on the result of part (d) and the shape of the stability region
of Backward Euler, what do you expect the stability region $S_N$ of part (c)
to converge to as $N\goto\infty$?

\end{enumerate}


% uncomment the next two lines if you want to insert solution...
%\vskip 1cm
%{\bf Solution:}

% insert your solution here!


%--------------------------------------------------------------------------
\vskip 1cm
\hrule
{\bf Problem 4}

\begin{enumerate}
\item
Write Python code to compute the exact solution
the linear system (7.10) modeling simple
reaction kinetics.  Reproduce Figure 7.4 by
using the same choice of parameters and initial conditions as specified in
Example 7.9.
\item Write code to solve the same problem with the Forward Euler method and
observe what happens if the time step $k$ is chosen to be the largest
possible value allowed by absolute stability. Check with smaller $k$ also to
verify that the method converges.
\item Change the reaction rates to $K_1=10, ~K_2=1$ and repeat the tests of
the previous part.
\item Implement the backward Euler method and show that it remains stable
(although perhaps not accurate) for arbitrarily large time steps.
\item Implement the TR-BDF2 method on this same problem and verify that it
is second-order accurate at time $T=2$ with $K_1=10, ~K_2=1$ and the same
initial conditions as before.
\end{enumerate} 


% uncomment the next two lines if you want to insert solution...
%\vskip 1cm
%{\bf Solution:}

% insert your solution here!



%--------------------------------------------------------------------------
\end{document}

